\chapter{Related Work}

% \textit{Note: Describe related work regarding your topic and emphasize your (scientific) contribution in \textbf{contrast} to existing approaches / concepts / workflows. Related work is usually current research by others and you defend yourself against the statement: ``Why is your thesis relevant? The problem was already solved by XYZ.'' If you have multiple related works, use subsections to separate them.}

\section{Fine Motor Decline and Psychomotor Impairment Detection}

Similar research has been done in the work of Kapsecker et al \cite{kapsecker2022}. 
In this research, data of typing behaviors are also accumulated, such as typing speed and variation in character usage. 
Kapsecker's research, however, used a modified version of the iOS default keyboard to able to gather these data. 
This modified version of the iOS keyboard brings forwards a limitation in this research, specifically that it shows deviation from the standard keyboard which cause more frequent use of backspace due to typos and different typing behaviors in general.
Since the default keyboard of iOS devices is highly optimized, even the slightest changes could affect the user significantly. 
Differences in structure and layout from this default keyboard, however minor, could cause noticeable changes in the behavior of the users and thus the gathered data

This research has three main findings. 
The first finding show that the uniform statistical property can be found in the subjects' typing patterns, i.e. their typing speed and their associated overall distribution. 
The second finding is also regarding the typing speed. 
The results of the research shows that there is a strong consistency in typing speed between healthy subjects regardless of potential impact factors, such as daytime. 
This implies that the method of recording typing behavior, in this case with a custom keyboard, is suitable for measuring baseline deviations for both short and long term. 
The third finding shows that there is a high correlation of approximately 0.8 between frequency and average transition time. 
It implies that subjects show different transition time during typing characters that are rarely used and more often used. 
The system used in this research seems sensitive enough to notice these differences.

These findings suggest that it is possible to detect cognitive and psychomotor impairments through recording and analyzing typing behavior.
In the effort of extending this research, the author is hoping to achieve similar results while decreasing the limitations, specifically the limitation caused by using custom keyboard.

Van Waes et al. suggested in 2017 that typing tasks might provide a more accessible alternative for both patients and clinicians. 
Additionally, the research explores the use of keystroke dynamics as digital biomarkers, which could enhance the diagnostic accuracy for detecting fine motor decline associated with neuropsychiatric disorders \cite{VanWaes2017}. 
The research highlights how these tasks could serve as a valuable tool in assessing typing and motor skills, which may decline in patients with Alzheimer's disease.

Mastoras et al. suggested in their research in 2019  that typing patterns can be indicative of psychomotor impairment associated with depressive tendencies \cite{Mastoras2019}. 
This research contributes to the development of unobtrusive, high-frequency monitoring tools for depressive tendencies, providing a potential method for early detection and intervention in everyday settings. 
The findings highlight the potential of using everyday interactions with mobile devices as a source of data for mental health monitoring.

A newer research in 2023 by Tripathi et al. showed the recognition of neurodegenerative diseases, such as \ac{PD}, using typing patterns is an emerging field that leverages keystroke dynamics \cite{Tripathi2023}. 
This approach involves analyzing the time it takes for individuals to press and release keyboard keys during typing, known as hold time, as well as the time between keystrokes, referred to as flight time. 
These metrics can be used to detect signs of \ac{PD} in an ecologically valid setup, such as at the subject's home.

These researches highlighted the possibility of detecting fine motor decline and psychomotor impairment through typing pattern on a keyboard. 
They showed that through analysing keystroke dynamics, flight time and hold time, psychomotor impairment can be detected. 

\section{Fine Motor Decline and Psychomotor Impairment Detection}

In an article published in 1984, Salthouse revealed some interesthing insights on his research on the effect of age on typing pattern \cite{Salthouse1984}.
Older typists are generally not slower in overall typing speed compared to younger typists.
However, older typists show slower performance in some areas: tapping rate (the speed of repetitive finger movements) and choice reaction time (the time between a stimulus and an action).
Older typists tend to compensate this by being more conscious about characters farther ahead in the text they are typing.
This in turn makes it so that older typists are able to maintain a more constant typing speed compared to their younger counterparts.

TODO: add more research on this topic

To the researcher knowledge, there has not been a study about using typing behavior on a mobile optimized application to infer the age group of a user.