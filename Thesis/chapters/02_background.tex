\chapter{Background}

% \textit{Note: Describe each proven technology / concept shortly that is important to understand your thesis. Point out why it is interesting for your thesis. Make sure to incorporate references to important literature here.}

\section{Mobile-Optimized Applications and Accessibility for Elderly Users}

It is crucial that the application is both mobile-optimized and also accessible to elderly users.
Since the author wants to use data of people typing on their smartphones, the application needs to be mobile-optimized.
Mobile-optimized application ensures that the data gathered by this application captures typing pattern of its users correctly.
Irregularities, such as typing mistakes or longer typing interval, should not be caused by the difficulty of using this application.
Instead this irregularities should reflect human error, that most likely to happens more frequently with older subjects.
Among other thigs, the \ac{UI} design of the application should be clean and minimalist, focusing on essential features and contents.
Unnecessary elements and visual clutters should be removed to create an intuitive \ac{UI} that is easy to navigate on smaller screens.
Since there are many types and sizes of smartphones, it is also important to build a responsive web application that is able to adjust to common screen sizes.

The application also need to be accessible for elderly users.
This is done to prevent a false positive condition, where significantly more typing errors happen on elderly subjects because of the difficulty of using the application.
Designing an accessible \ac{UI} for the elderly requires some considerations as suggested by Gomez-Hernandez et. al. in their research in 2023\cite{hernandez2023}:
\begin{enumerate}
    \item \textbf{Bigger Interactive Elements:} The size of interactive elements, such as buttons and forms, need to be bigger to make them more accessible.
    This improvement could help with possible vision or motoric impairments.
    \item \textbf{High contrast:} Another method to help with vision impairment.
    This could help increase readability.
    \item \textbf{Font Selection:} The font used should be easily readable, especially on smaller screens.
    \item \textbf{Spacing:} Enough spacing should be added to improve readability und user experience.
    \item \textbf{Apropriate color choices}: Avoiding colors like blue, violet, and green, and yellow, which can become harder to distinguish with age due to shifts in color perception.
    \item \textbf{Centralized content}: Placing key elements in the center of the interface to make them easier to find and interact with.

\end{enumerate}


\section{Backend Architecture: Java Spring Boot}
On the backend side of the application, Java Spring Boot framework is utilised to manage and process data that are sent by the frontend of the application.
The Spring Framework is an application framework and inversion of control container for the Java programming language.
Spring Boot is an open-source Java framework for applications based on Spring to help project startup and management easier.
This framework provides libraries that help to develop a scalable, maintainable and secure web applications.
It is important to develop the application in this manner so that new features and improvement could easily be added during or after the writing of this thesis.
If the occassion arises, this application would also be ready to be reused for a possible follow-up research on this topic.

Spring Boot provides functionalities that help to ensure clean architecture and adherence to coding principles, such as DRY (Don't Repeat Yourself).
Another important feature provied by Spring Boot is RESTful services, that support separation of the client and the server.
Other than aiding on building a clean architecture, RESTful services also further ensure the possibility of reusing the code for further researches.
This application uses \ac{DAO} and \ac{DTO} patterns to manage data efficiently.
This separation of concerns between \ac{DAO} and \ac{DTO} adheres to clean architecture principles, improving maintainability and testability.

\section{Database Management: PostgreSQL}
PostgreSQL is an open-source, object-relational database system known for its scalability, reliability and support for complex queries.
These attributes make PostgreSQL ideal for storing a huge amount typing data and keystroke logs.
Features such as JSONB data type that is offered by PostgreSQL also help to simplify data processing and storing, especially data of keystroke logs. 
Another reason why PostgreSQL is used in this project is its indexing and search capabilities.
This will be crucial for fast retrieval of user data, allowing efficient typing pattern analysis that would be useful if instant analysis feature is ever built.

\section{Large Language Models: Llama3}
The Llama3 is the latest \ac{LLM} develop by Meta inc.
In addition to Llama3's impressive performance compared to other \ac{LLM}s, Llama3 is multilingual and can support both english and german language well.
This makes Llama3 the perfect language model to use for this thesis.
In this thesis, Llama3 will be used to chat with the users in real-time.
The language model is prompted to try to get as much response from the users as possible, so that the typing data can be collected.

It is interesting to see whether \ac{LLM}s such as Llama3 would be able to also analyze the users' typing data.
It can potentially help identify changes and irregularities in typing pattern that might indicate the age of the user.
If this is possible, a real-time analysis of the typing pattern might be able to be implemented with the help of language models.
This topic is, however, not in the scope of this thesis and is only a possible future research.

\section{Containerization: Docker}
To further promotes the scalability of the application, the author utilizes Docker.
Docker is a platform for delivering software in packages called containers.
Containerization is an important part of deploying a scalable application across various environments.
Docker allows applications to be deployed separately (frontend, backend, database and \ac{LLM}) in their own separated containers.
This makes it easier for the application to be reused and rebuild in other environments, such as on the university servers.

\section{Pattern Recognition and Analysis of Typing Pattern}
To recognize typing pattern of users from different age groups, a statistical analysis of the typing pattern will be carried out.

TODO: find and explain statistical analysis correlation methods

Metrics such as keystroke intervals, typing speed, and error rates will be analyzed to distinguish typing patterns between each age groups. 

TODO: find correlation between typing metrics and age group based on the statistical analysis.

TODO: find pattern recognition method or algo such as Hidden Markov Models (HMM), Dynamic Time Warping (DTW), or Neural Networks 

\section{Security and Privacy in Data Collection}
TODO: do we need this part? Privacy is not as important since the data is not from patients. Maybe explain about anonimity
