\chapter{Introduction}

% \textit{Note: Introduce the topic of your thesis, e.g. with a little historical overview.}

\ac{AD} and \ac{PD}, are chronic and progressive neurodegenerative diseases that primarily affect the nervous system.
This diseases lead to the degeneration of motoric and cognitive abilities.
These diseases are often irreversible and incurable, posing significant public health challenges.
As populations age, the risk of such disorders increase substantially, making early detection crucial.

Historically, the diagnosis of neurodegenerative diseases has been done throuh clinical observations, imaging, and biomarkers.
Even though medical technologies have improved significantly over the years, there are still many challenges for early diagnosis.
One of the main reason for this is because neurodegenerative diseases develop gradually over time.
Early symptoms of these diseases can easily be overlooked or mistaken as normal aging \cite{manera2023}.
As a result, these early symptons are often ignored until the disease itself has reached a critical stage, where the chance of treatment declines significantly. 
Furthermore, current diagnostic methods are expensive, invasive, and often inaccessible to a large portion of the population.

A better method is obviously needed to battle these insidious diseases.
The rise of technologies such as smartphones open up new possibilities for early detection of these conditions.
One example of such possibility is to utilise a phone application as a mean to detect early neurodegenerative diseases.
Studies have shown that one of the effect of these diseases, i.e. impairments of motoric functions, will be reflected on how a person types \cite{mcisaac2023}.
This changes in typing behaviors, such as typing speed, accuracy, and keystroke patterns can then be recorded by the application.
The data acquired by this application might be able to be used to analyse early symptons of \ac{AD} and \ac{PD}.

\section{Problem}

% \textit{Note: Describe the problem that you like to address in your thesis to show the importance of your work.
%  Focus on the negative symptoms of the currently available solution.
% }

It is clear from the facts mentioned above, that neurodegenerative diseases are problems that need to be addressed.
World Health Organisation estimates that there are approximately 50 million people worldwide affected by these disorders.
Most of the sufferer are elderly, since age is one of the main risk factor.
As the most common neurogedenerative disorder, \ac{AD} still lacks an effective cure.
It is even harder to treat the more progressive the disease progress.
That is why the importance of an effective way to diagnose the disease early cannot be overstated. 
In the current state, however, misdiagnosis rates are still high, reaching up to 20\%.
Not only that, current diagnostic methods are either invasive, costly, or impractical for widespread use.

Similarly, \ac{PD}, the second most common neurodegenerative disorder, has no cure and limited treatment options. 
For this disease too an effective mean for early diagnosis is of utmost importance.
Since with a successful early detection, the progression of the disease can be slowed significantly, improving the quality of life of patients greatly.
The current diagnostic methods for this disease, however, rely mostly on the observation of changes of motoric symptoms.
These methods, as previously discussed, are unreliable for many reasons.

In both conditions, early intervention can significantly improve patient outcomes, but existing diagnostic tools fail to provide a practical and accurate solution for early detection. 
There is a pressing need for non-invasive, cost-effective, and widely accessible methods to detect early signs of neurodegenerative diseases before the onset of significant symptoms.

\section{Motivation}

% \textit{Note: Motivate scientifically why solving this problem is necessary.What kind of benefits do we have by solving the problem?}

The motivation for this thesis comes from the urgent need to find better diagnostic methods for neurodegenerative diseases. 
Finding methods to effectively detect early these diseases would significantly improve general public health.
Early detection allows for earlier interventions, which will then slow the progression of the diseases.
This would improve treatment outcomes, and ultimately reduce the burden on healthcare systems.

From a scientific point of view, the research on using digital biomarker, i.e. typing pattern, as a mean to detect neurodegenerative diseases is underexplored.
This research could give insights into how effective common tools and activities can be used to improve public health.
Typing has became a common daily activity for most modern human, especially with the widespread use of smartphones and chat applications.
This means that it can be a low-cost and non-invasive method to detect subtle motoric or cognitive impairments.
In the most ideal case, the subject would not even notice that they are being monitored for these diseases and can go on about enjoying their daily lives.

Taking advantage of these common daily activities could help reduce the risk of neurodegenerative diseases, by diagnosing them as early as possible.
Furthermore, these methods would also be accessible to people that lives in less developed countries with less developed medical technologies.
This would ensure equal chances to fight against these neurodegenerative diseases.
By developing a mobile-optimized web application that can collect and analyze typing data in real time, it would also become easier to monitor people's condition.
Especially the condition of those that are more prone to this diseasesm, i.e. the elderly.

\section{Objectives}

% \textit{Note: Describe the research goals and/or research questions and how you address them by summarizing what you want to achieve in your thesis, e.g. develping a system and then evaluating it.}

Developing a web-based chat application that can capture and analyze typing behavior for early detection of neurodegenerative diseases is an ambitious goal.
That goal is unfortunately not in the scope of this thesis.
Collecting pathological data that is required for the aforementioned goal requires a lot of process and time, which does not align with the time limitation of this thesis.
Another more suitable objective for this thesis would be to explore whether it is possible to determine the age of a user based on their typing patterns.
The author believe that this thesis will pave the way for a more advanced research on this matter.
This thesis wants to show that typing pattern can be used to identify the characteristics of the user, in this case, the age group.

Specifically, this thesis aims to:
\begin{enumerate}
    \item Develop mobile-optimized chat application that collects the user's typing data.
    The main focus is to collect samples from individuals across different age groups.
    \item Practice clean architecture and secure coding practices to make sure the application is reliable, scalable, and maintainable. 
    The chat application will be designed to be user-friendly for all age groups, specifically the elderly.
    \item Analyze the gathered typing behavior data to identify patterns or statistical distributions that may correlate with the user's age. 
    Metrics such as typing speed, keystroke intervals, and error rates will be examined to determine if they can give indication of the user's age group.
    \item Evaluate the accuracy of using typing behavior as a predictor of age.
    The identified patterns need to be consistent enough to be able to reliably be used to estimate the user's age group.
\end{enumerate}

The goal of this thesis is to research the feasibility of using data of typing behavior gathered by the application to profile the user in an age group.
If this is achieved, the author hopes that this could give insights that would be valuable for future research in user profiling or cognitive assessments.
The application could also be further developed to be able to analyse more complex matters, such as early signs of neurodegenerative disorders.
Another possible improvement would be adding real-time analysis of the typing pattern and integration with healthcare systems.
This would be beneficials for patients and clinicians alike.