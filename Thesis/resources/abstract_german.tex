%abstract german
% TODO -> check translation
Neurodegenerative Krankheiten sind chronische Erkrankungen, die im Laufe der Zeit Teile des Nervensystems der Betroffenen, insbesondere das Gehirn, zerstören und schädigen. 
Diese Krankheiten stellen eine große Herausforderung für die allgemeine öffentliche Gesundheit dar, da die Schäden dauerhaft und unheilbar sind. 
Sie treten vor allem bei älteren Menschen auf, da das Älterwerden der größte Risikofaktor ist. 
Darüber hinaus ist die Früherkennung dieser Krankheiten ineffizient, unpraktisch und hat nur einen verschwindend geringen Erfolgsanteil. 
Es besteht ein Bedarf an besseren Erkennungsmethoden, die kostengünstig, benutzerfreundlich und genau sind.

Ziel dieser Arbeit ist es, den Weg für die Entwicklung besserer Erkennungsmethoden zu ebnen.
In dieser Arbeit wird eine Lösung vorgeschlagen, die die Entwicklung einer für Mobilgeräte optimierten Webanwendung beinhaltet, um Tippdaten von Benutzern verschiedener Altersgruppen zu sammeln. 
Es wird eine saubere und robuste Architekturstruktur verwendet, um Zuverlässigkeit, Skalierbarkeit und Wartbarkeit zu gewährleisten. 
Es sollte auch sichergestellt werden, dass die Anwendung in der Lage ist, die gesammelten Daten effektiv zu verarbeiten und zu speichern, so dass die Daten für Forschungszwecke verwendet werden können.
Die Anwendung kann dann auch mit erweiterten Funktionen weiterentwickelt werden. 
Ein Beispiel für eine solche zusätzliche Funktion wäre ein Analysebereich, in dem die Daten zum Tippverhalten einer Person mit einem Klick auf eine Schaltfläche sofort analysiert werden können.

Die Analyse der Daten wird mit dem Ziel durchgeführt, mathematische Eigenschaften zu finden.
Diese mathematischen Eigenschaften können dann verwendet werden, um jeden Nutzer in die entsprechenden Altersgruppen einzuteilen.
Das Hauptaugenmerk dieser Arbeit liegt auf der Frage, ob bestimmte Biomacher, wie z.B. das Tippverhalten, zur Unterscheidung von Merkmalen einer Person verwendet werden können.
Die aus dieser Arbeit abgeleiteten Schlussfolgerungen könnten Aufschluss darüber geben, inwieweit Biomarker zur effektiven Überwachung des Gesundheitszustands des Nutzers eingesetzt werden können.
Insbesondere hofft der Autor, dass diese Erkenntnisse für die Forschung zur Erkennung früher Anzeichen von neurodegenerativen Erkrankungen mit einer einfachen Methode zur Erfassung von Tippmusterdaten von Nutzen sein könnten.