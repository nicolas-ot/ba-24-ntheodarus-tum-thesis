%abstract german
% TODO -> check translation
Neurodegenerative Krankheiten sind chronische Erkrankungen, die im Laufe der Zeit Teile des Nervensystems der Betroffenen, insbesondere das Gehirn, zerstören und schädigen. 
Diese Krankheiten stellen eine große Herausforderung für die allgemeine öffentliche Gesundheit dar, da die Schäden dauerhaft und unheilbar sind. 
Sie treten vor allem bei älteren Menschen auf, da das Älterwerden der größte Risikofaktor ist. 
Darüber hinaus ist die Früherkennung dieser Krankheiten ineffizient, unpraktisch und hat nur einen verschwindend geringen Erfolgsanteil. 
Es besteht ein Bedarf an besseren Erkennungsmethoden, die kostengünstig, benutzerfreundlich und genau sind.

Ziel dieser Arbeit ist es, den Weg für die Entwicklung der oben genannten besseren Erkennungsmethoden zu ebnen.
In dieser Arbeit wird eine für Mobilgeräte optimierte Web-Applikation zur Erfassung von Tippdaten von Nutzern verschiedener Altersgruppen entwickelt. 
Es wird eine saubere und robuste Architekturstruktur verwendet, um Zuverlässigkeit, Skalierbarkeit und Wartbarkeit zu gewährleisten. 
Es soll auch sichergestellt werden, dass die Anwendung in der Lage ist, die gesammelten Daten effektiv zu verarbeiten und zu speichern, so dass die Daten für Forschungszwecke verwendet werden können.
Nach dieser Arbeit kann die Anwendung hoffentlich in der Zukunft mit fortgeschritteneren Funktionen weiterentwickelt werden. 
Ein Beispiel für eine solche zusätzliche Funktion wäre eine Analysefunktion, mit der die Tippverhaltensdaten einer Person mit einem Mausklick sofort analysiert werden können.

Die Analyse der Daten wird mit dem Ziel durchgeführt, statistische Eigenschaften zu finden.
Diese statistischen Eigenschaften können hoffentlich genutzt werden, um zwischen verschiedenen Altersgruppen zu unterscheiden.
Der Hauptbeitrag dieser Arbeit besteht darin, eine robuste Anwendung zu entwickeln, die Typisierungsdaten von ihren Nutzern sammeln kann, und diese Daten dann zu analysieren, um interessante Eigenschaften zu finden. 
Die aus dieser Arbeit abgeleitete Schlussfolgerung könnte einen Einblick in die Durchführbarkeit der Verwendung von Biomarkern zur effektiven Überwachung des Gesundheitszustands des Benutzers geben.
Insbesondere hofft der Autor, dass diese Erkenntnisse für die Forschung zur Erkennung früher Anzeichen von neurodegenerativen Erkrankungen mit einer einfachen Methode zur Erfassung von Tippmusterdaten von Nutzen sein könnten.