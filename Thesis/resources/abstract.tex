%abstract english

% \textit{Note: you will write two abstracts for your thesis}

% \textit{
% 	\textbf{the first abstract} will be written at the very beginning of your thesis.
% 	It shall contain a description of the problem, you want to solve, and present a plan how you plan to address that problem.
% }

% \textit{
% 	The purpose of this first abstract is to specify the topic of the thesis.
% 	It is meant to reflect to the supervisor, how you have understood the task.
% 	Also, it gives you the opportunity to define a focus or to add topics, that you are interested in, to the scope of your thesis.
% }

% \textit{\textbf{1. paragraph:} What is the motivation of your thesis? Why is it interesting from a scientific point of view? Which main problem do you like to solve?}

% \textit{\textbf{2. paragraph:} What is the purpose of the document? What is the main content, the main contribution?}

% \textit{\textbf{3. paragraph:} What is your methodology? How do you proceed?}

Neurodegenerative diseases are chronic conditions that destroy and damage part of nervous system of the sufferer over time, especially the brain. 
This diseases pose a significant challenge for general public health, since the damages are permanent and incurable. 
This condition happens mainly on elderly people, given that aging is the greatest risk factor. 
Moreover, early detection of these diseases are inefficient, impractical and only have minuscule success percentage. 
There is a need for better detection methods that are cost-effective, user-friendly and accurate.

This thesis aims to pave the way of developing the aforementioned better detection methods.
This thesis proposes a solution that involve developing a mobile optimized web application to gather typing data from users of different age groups. 
A clean and robust architecture structure is utilised to guarantee reliability, scalability and maintainability. 
It should also be ensured that the application is able to effectively process and save the collected data, so that the data can be used for research purposes.
The application can also then be developed further with more advanced features. 
An example of such additional feature would be an analysis section, where typing behaviour data of a person can instantly be analysed with a click of a button.

An analysis of the data will be performed with the goal to find mathematical properties.
This mathematical properties can then be used to categorize each user into their corresponding age groups.
Understanding whether certain biomakers, e.g. typing pattern, can be used to differentiate characteristics of a person is the main focus of this thesis.
The conclusion derived from this thesis could give insight into the feasibility of utilising biomarkers to effectively monitor health condition of the user.
Specifically, the author hopes that this findings would be beneficials for research on detecting early signs of neurodegenerative disorders effectively with a simple method of collecting typing pattern data.

